%\RequirePackage{lineno}
\documentclass[aps,prc,preprint,superscriptaddress,showpacs,showkeys]{revtex4-1}
\usepackage{graphicx}

\begin{document}

%\linenumbers

\title{{\Large $J/\psi$ suppression and regenration in PbPb collisions at $\sqrt s_{NN}$ =  2.76 TeV }}

\author{\large V. Kumar}
\author{\large P. Shukla}
\email{pshukla@barc.gov.in}
\affiliation{Nuclear Physics Division, Bhabha Atomic Research Center, Mumbai, India}
\affiliation{Homi Bhabha National Institute, Anushakti Nagar, Mumbai, India}

\date{\today}


\begin{abstract}

Here we summarize $J/\psi$ suppression and regenration in quark gluon plasma. 
We assume that thermal activation due to gluons is main mechanism for $J/\psi$
break-up.
 

\end{abstract}

\pacs{12.38.Mh, 24.85.+p, 25.75.-q}

\keywords{quark-gluon plasma, kinetic equation}


\maketitle


\section{Introduction}

%%%% Taken from Upsilon ratio paper  %%%%%%%%%%%

   The heavy ion collisions produce matter at extreme temperatures and densities where 
it is expected to be in the form of Quark Gluon Plasma 
(QGP), a phase in which the quarks and gluons can move far beyond  the size of a nucleon 
making color degrees of freedom dominant in the medium. 
  The experimental effort to produce such matter started with low energy CERN accelerator 
SPS and evolved through voluminous results 
from heavy ion collision at Relativistic Heavy Ion Collider (RHIC) \cite{INTRO}.
The recent results from Large Hadron Collider (LHC) experiments \cite{QGP_Tc} are 
pointing towards formation of high temperature system in many ways similar to the matter
produced at RHIC. 
  One of the most important signal of QGP is the suppression of 
quarkonium states \cite{SATZ}, both of the charmonium ($J/\psi$, $\psi(2S)$, $\chi_{c}$, etc) 
and the bottomonium ($\Upsilon(1S)$ , $\Upsilon(2S)$, $\chi_{b}$, etc) families. This is thought to be a 
direct effect of deconfinement, when the binding potential between the constituents of a quarkonium state, 
a heavy quark and its antiquark, is screened by the colour charges of the surrounding light quarks and gluons. 
 The ATLAS and CMS experiments have carried out detailed quarkonia measurements in Pb+Pb collisions 
with the higher energy and luminosity available at the LHC.
 The ATLAS measurements \cite{ATLAS} show suppression of inclusive $J/\psi$ with high transverse momenta $p_T$  
in central PbPb collisions compared to peripheral collisions at $\sqrt s_{NN} = 2.76$ TeV. 
  Similarly, CMS measured a steady and smooth decrease of suppression 
of prompt $J/\psi$ as a function of centrality with nuclear modification factor $R_{AA}$ remaining $<$ 1 even 
in the peripheral bin \cite{JCMS}. 
 
 The melting temperature of the quarkonia states depends on their binding energy. The ground states, 
$J/\psi$ and $\Upsilon(1S)$ are expected to dissolve at significantly higher temperatures than the 
more loosely bound excited states. 
 
  The $\Upsilon(2S)$ and $\Upsilon(3S)$ have smaller binding energies as compared to ground
state $\Upsilon(1S)$ and hence are expected to dissolve at a lower temperature. 
 With the 2011 Pb+Pb run the CMS published results on sequential suppression of 
$\Upsilon(nS)$ states as a function of centrality \cite{CMSU2} with enlarged statistics
over their first measurement \cite{UCMS}, where a suppression of the excited $\Upsilon$ states with respect to the ground state have been observed  
in PbPb collisions compared to pp collisions at $\sqrt s_{NN} = 2.76$ TeV.\cite{YSuppAbdShuk}

  The quarkonia yields in heavy ion collisions are also modified due to non-QGP effects such as
shadowing, an effect due to change of the parton distribution functions inside the nucleus,
and dissociation due to nuclear or comover interaction \cite{Vogt}.


  If large number of heavy quarks are produced in initial heavy ion collisions at LHC energy 
this could even lead to enhancement of quarkonia via statistical recombination \cite{Rapp1,Rapp2}. 
   
 In this paper, we calculate the charmonia suppression due to thermal gluon activation  in an expanding
QGP. Another sources which can alter the yield of charmonium are considered e.g. nuclear shadowing 
and gluon saturation. Impact parameter dependence of parton distribution functions is also taken in account
and at last we consider the effect of J/$\psi$ regeneration using the statistical hadronization model.
J/$\psi$ yield is studied as a function of impact parameter or centerality of events. Calculations are compared
with experimental data, where data is available.



\section{Charm production rates}

  
  The large heavy quark mass allows their production to be calculated in 
perturbative QCD.  We calculate the production cross sections for 
$c\overline c$ and  $b\overline b$ pairs to NLO in pQCD \cite{GAVIN,LIN} using
the CTEQ6M parton densities \cite{CTEQ6}.  The central EPS09 parameter set 
\cite{EPS09} is used to calculate the modifications of the parton densities in 
Pb+Pb collisions.  
 
 We use the same set of parameters
as that of Ref.~\cite{CNV} with the exclusive NLO calculation of Ref.~\cite{MNR}
to obtain the exclusive $Q \overline Q$ pair rates as well as their decays
to dileptons.  We take $m_c = 1.5$~GeV/$c^2$, $\mu_F/m_T = \mu_R/m_T = 1$ and  
$m_b = 4.75$~GeV/$c^2$, $\mu_F/m_T = \mu_R/m_T = 1$ as the central values for
charm and bottom production respectively.  Here $\mu_F$ is the factorization 
scale, $\mu_R$ is the renormalization scale and $m_T = \sqrt{m^2 + p_T^2}$.  
The mass and scale variations are added in quadrature to obtain the uncertainty
bands \cite{CNV} \cite{ContinuumVKShuk}.
    
  
  The production cross sections for heavy flavor at $\sqrt{s_{_{NN}}}= 2.76$ 
TeV are shown in Table~\ref{NLOcros}.  The number of $Q \overline Q$ pairs
in a minimum bias Pb+Pb event is obtained from the per nucleon cross
section, $\sigma_{\rm PbPb}$, by

\begin{eqnarray}
N_{Q \overline Q} = {A^2 \sigma_{\rm PbPb}^{Q \overline Q}  \over  
\sigma_{\rm PbPb}^{\rm tot}} \, \, .
\end{eqnarray}

At 2.76 TeV, the total Pb+Pb cross section, $\sigma_{\rm PbPb}^{\rm tot}$, 
is 7.65 b \cite{PbPbTotal}.

\begin{table}
\caption[]{Heavy flavor cross sections at 
$\sqrt{s_{_{NN}}}= 2.76$ TeV.  The cross sections are given per nucleon while
$N_{Q \overline Q}$ and $N_{l^+ l^-}$ are the number of $Q \overline Q$ and lepton 
pairs per Pb+Pb event.  The uncertainties in the heavy flavor cross section are
based on the Pb+Pb central values with the mass and scale uncertainties added
in quadrature.}
\label{NLOcros}
\begin{tabular}{c|c|c} 
\hline 
                 & $ c \overline c$     &J/$\psi$    \\
                 
\hline
$\sigma_{\rm PbPb}$   & $1.76^{+2.32}_{-1.29}$ mb       & $31.4$ $\mu$b \\
$N_{Q\overline Q}$      & $9.95^{+13.10}_{-7.30}$           & $0.177$     \\

\hline
\end{tabular}
\end{table}




\section{Shadowing of nuclear parton distribution function}

 It will be updated from ferrero paper.
 



\section{Nuclear Absorption}


It will be updated later.


\section{Statistical Hadronization Model}

  The heavy quark production at LHC is substantial which may lead to incoherent 
recombination of uncorrelated pairs of heavy quarks and anti quarks which result 
from multiple pair production. Two different approaches have been considered;
statistical hadronization and kinetic formation. 
In statistical approach \cite{MUNZI} the number of J/$\psi$ produced is given by 

  \begin{equation}
N_{J/\psi} = 4 {n_{ch} n_{J/\psi} \over n_{\rm open}^2}  {N_{c\bar c}^2 \over N_{ch} }.
\end{equation}
where $n_i$'s are the thermal densities and $N_{c\bar c}$ is the number of charm pairs produced 
and $N_{ch}$ is the number of total charged 
particle produced. 

The freeze out parameters are $T=170$ MeV and $\mu_B = 0$. For
$dN_{ch}/dy = 1600$ \cite{MULT} and $dN_{c \bar c} /dy = 17.8$, we obtain $dN_{J/\psi} /dy = ?$.
Now thermal densities can be calculated using thermal distributions of various particles.

\begin{equation}
n_{open} = {4\pi \,g_{D} \over (2\pi)^{3}} \int_{0}^{\infty} f_{D}(p)p^{2}dp 
\end{equation}

by the same method we can calculate the J/$\psi$ number density as
\begin{equation}
n_{J/\psi} = {4\pi \,g_{J/\psi} \over (2\pi)^{3}} \int_{0}^{\infty} f_{J/\psi}(p)p^{2}dp 
\end{equation}

N$_{c\bar{c}}$ = 17.8 as given in Table above. For total charge particle density We can use 

\begin{equation}
n_{ch} = n_{\pi^{+}} + n_{\pi^{-}} + n_{K^{+}} + n_{K^{-}} + n_{p} + n_{\bar{p}} 
\end{equation}

where
\begin{equation}
n_{\pi^{+}} = {4\pi \,g_{\pi} \over (2\pi)^{3}} \int_{0}^{\infty} f_{\pi{+}}(p)p^{2}dp
\end{equation}


and so on. 

%\int_{\tau_0}^{\tau_f}


\section{Kinetic Model}

  In Kinetic approach \cite{THEWS} the proper time evolution of the $J/\psi$ population is given by the rate equation 
\begin{equation}\label{eqkin}
{dN_{J/\psi} \over d\tau}  = \lambda_F {N_c N_{\bar c} \over V(\tau)} - \lambda_D N_{J/\psi} \rho_g.
\end{equation}

with $\rho_g$ the number density of gluons, $\tau$ the proper time
and $V(\tau)$ the volume of the deconfined spatial region.
The reactivities $\lambda_{F,D}$ are
the reaction rates $\langle \sigma v_{\mathrm{rel}} \rangle$
averaged over the momentum distribution of the initial
participants, i.e. $c$ and $\bar c$ for $\lambda_F$ and
$J/\psi$ and $g$ for $\lambda_D$.
The gluon density is determined by the equilibrium value in the
QGP at each temperature.  For simplicity, it is assumed to be
spatially homogeneous, as are the charm quark and $J/\psi$ distributions.

We enforce exact charm conservation in solving this equation, but
in practice this means that the number of charm quarks $N_c$
and anticharm quarks $N_{\bar{c}}$ are always approximately equal
to the number of initial pairs $N_{c\bar{c}}$ for the following reasons:
(a) Reactions in which charm quark-antiquark pairs
annihilate into light quarks or gluons are small since the charm
density due to initial charm is less than the thermal equilibrium
value during most of the time; (b) Production of additional
charm quark pairs from interactions of light quarks and gluons is
negligible during the time of deconfinement \cite{letessier};
(c) Formation of other states in the charmonium spectrum is a small
fraction of initial charm (as is the fraction of $J/\psi$ itself);
and (d) Disappearance of single charm quarks or antiquarks via
formation of open charm mesons is effectively reversed immediately
because the time scale for dissociation of these large states 
with small binding energy is typically less than a fraction of 
a fermi.The expansion is taken to be isentropic,
\begin{equation}
 VT^{3} = {\mathrm constant}
 \end{equation}
  
 which then provides a generic temperature-time profile.


It is evident that the solution of Equation \ref{eqkin} grows quadratically
with initial charm $N_{c\bar{c}}$, as long as the total $J/\psi \ll N_{c\bar{c}}$.  In
this case we can write an analytic expression

\begin{equation}
N_{J/\psi}(\tau_f) = \epsilon(\tau_f) \,N_{J/\psi}(\tau_0)
+\epsilon(\tau_f)\,{N_{c\bar{c}}^2}\int_{\tau_0}^{\tau_f}
{\lambda_{\mathrm{F}} \over V(\tau)\,\epsilon(\tau)} d\tau
\label{eqbeta}
\end{equation}



where $\tau_f$ is the hadronization time determined by the
initial temperature ($T_0$ )  and
final temperature ($T_f$ ).
The function 
\begin{equation}
\epsilon(\tau_f) = e^{-\int_{\tau_0}^{\tau_f}{\lambda_{\mathrm{D}}\,\rho_g\,d\tau}} 
\end{equation}
Now for a longitudinal isentropic expansion,
\begin{equation}
V(\tau)=V_o\,{\tau \over \tau_o}
=\pi\,R^{2}\tau_0\,{\tau \over \tau_o}
=\pi\,R^{2}\tau
\end{equation}

it gives 

\begin{equation}
N_{J/\psi}(\tau_f) = \epsilon(\tau_f)  N_{J/\psi}(\tau_0) 
                  + \epsilon(\tau_f) {N_{c\bar{c}}^2 \over \pi R^{2}} \int_{\tau_0}^{\tau_f}
{  {\lambda_{\mathrm{F}} \over \tau\,\epsilon(\tau)} d\tau}
\label{eqbeta1}
\end{equation}

\begin{equation}
R_{AA} = \epsilon(\tau_f) + \epsilon(\tau_f) {N_{c\bar{c}}^2 \over \pi R^{2}\,N_{J/\psi}(\tau_0)} \int_{\tau_0}^{\tau_f}
{  {\lambda_{\mathrm{F}} \over \tau\,\epsilon(\tau)} d\tau}
\label{eqbeta1}
\end{equation}



R is the spatial extension of quark gluon plasma. 
%%% Centrality dependence from Upsilon ratio paper

\begin{equation}\label{rnpart}
R(N_{\rm part}) = R_0 \, \sqrt{N_{\rm part} \over N_{\rm part0} }.
\end{equation}
The initial temperature as a function of centrality is calculated by 
\begin{equation}\label{InT1}
T(N_{\rm part})^3 = T_0^3 \, \left({dN/d\eta \over N_{\rm part}/2}\right) / \left({dN/d\eta \over N_{\rm part}/2}\right)_{0-5\%},
\end{equation}

 where $T_0$ is the initial temperature assumed in 0-5\% centrality and $(dN/d\eta)$
is the multiplicity as a function of number of participants measured by ALICE experiment \cite{MULT}. 
Both ALICE and CMS \cite{CMSmult} measurements on multiplicity agree well with each other.

The initial temperature $T_0$ for 0-5\% central collisions for a given initial time $\tau_0$ is obtained by 

\begin{equation}\label{InT2}
T_{0}^{3}\tau_{0} = \frac{3.6}{4a_{q}\pi R_{0-5\%}^{2}}\left(\frac{dN}{d\eta}\right)_{0-5\%},
\end{equation}

Here $(dN/d\eta)_{0-5\%}$ = 1.5$\times$1600 obtained from the charge particle multiplicity measured in 
Pb+Pb collisions at 2.76 TeV \cite{MULT} and $a_{q}$ = 37$\pi^{2}$/90 is the degrees of freedom we take in 
in quark gluon phase. Using Eq.~(\ref{rnpart}) we can obtain the transverse size of the system
for 0-5\% centrality as $R_{0-5\%}$ = 0.92$R_0$. 
 For $\tau_{0}$ = 0.1 fm/$c$, we obtain $T_{0}$ as 0.62 GeV using Eq.~(\ref{InT2}). 
The critical temperature is taken as $T_{C}$ = 0.160 GeV \cite{QGP_Tc}. 


N$_{J/\psi}(\tau_0)$ and N$_{c\bar{c}}$ in Eq.~(\ref{eqbeta1}) will also be a function of 
centrality. They will scale as number of collision in particular centrality class.
Number of collisions and number of participants can be calculated from Glauber model 
calculations.

%%%%%%%%%%%%%%%%%%%%%%%%%%%%%%%%%%%%%%%%%%%%%%%%%%%%%%%%%%%%%%%%%%%%%%%%%%%%%%%%%%%%%%%%%%%%%%
%%%%%%%%%%%%%%%%%%%%%%%%%%%%%%%%%%%%%%%%%%%%%%%%%%%%%%%%%%%%%%%%%%%%
\subsection{Dissociation Rate}
The operator product expansion allows one to express the
hadron-$J/\psi$ inelastic cross section in terms of the
convolution of the gluon-$J/\psi$ dissociation cross section
with the gluon distribution inside the hadron \cite{ks94}.
The gluon-$J/\psi$ dissociation cross section is given by \cite{ks95}
\begin{equation}
  \sigma (q^0)= \frac{2\pi}{3}\left(\frac{32}{3}\right)^2
  \left(\frac{16\pi}{3g_s^2}\right)\frac{1}{m^2_Q}
  \frac{(q^0/\epsilon_0-1)^{3/2}}{(q^0/\epsilon_0)^5}\; , \label{eq1}
\end{equation}
where $g_s$ is the coupling constant of gluon and $c$ quark, $m_Q$
the $c$ quark mass, and $q^0$ the gluon energy in the $J/ \psi$ rest
frame;  its value must be larger than the $J/\psi$ binding energy
$\epsilon_0$. Since
for the tightly bound ground state of quarkonium
the binding force between the heavy quark and antiquark
is well approximated by the one-gluon-exchange Coulomb potential,
the $Q\bar{Q}$ bound state is hydrogen-like and the Coulomb relation
holds,
\begin{equation}
\epsilon_0 =\left({3g_s^2 \over 16\pi}\right)^2 m_Q \; . \label{eq2}
\end{equation}
The cross section thus can be rewritten as
\begin{equation}
  \sigma (q^0)=\frac{2\pi}{3}\left(\frac{32}{3}\right)^2
  \frac{1}{m_Q(\epsilon_0 m_Q)^{1/2}}
\frac{(q^0/\epsilon_0-1)^{3/2}}{(q^0/\epsilon_0)^5}\; . \label{eq3}
\end{equation}


As shown in Monte Carlo simulations \cite{KEXW}, the parton
density in the early stage of high-energy heavy-ion collisions
has an approximate Bjorken-type \cite{bj83} scaling behavior. We
will only consider $J/\psi$ suppression in the central rapidity
region ($y_{J/\psi}\simeq 0$). In this case, the $J/ \psi$ will move
in the transverse direction with a four-velocity
\begin{equation}
u=(M_T, \vec{P_T}, 0)/M_{J/\psi}, \label{eq4}
\end{equation}
where $M_T=\sqrt{P_T^2+M^2_{J/ \psi}}$ is defined as the $J/\psi$'s
transverse mass. A gluon with a four-momentum $k=(k^0,\vec{k})$
in the rest frame of the parton gas has an energy $q^0=k\cdot u$
in the rest frame of the $J/\psi$. 
\begin{eqnarray}
 q^{0} &= &\frac{k^{0}\,m_{T} + \vec{k} \cdot \vec{p_{T}}}{M_{J/\psi}}
\end{eqnarray}
in general q$^{0}$ can be written as
\begin{eqnarray}
 q^{0} &= &\frac{s-M_{J/\psi}^{2}}{M_J/\psi}
\end{eqnarray}
The thermal gluon-$J/\psi$ dissociation cross section is then defined as
\begin{eqnarray}
\langle v_{\rm rel} \sigma (k \cdot u)\rangle_k &= &\frac{\int d^3k v_{\rm rel} \sigma (k \cdot u) f(k^0;T)}{\int d^3k f(k^0;T)} \\
                                                &= &\frac{\int k^{2} dk dcos\theta d\phi v_{\rm rel} \sigma (k \cdot u) f(k^0;T)}{\int d^3k f(k^0;T)} \\
 \label{eq5}
\end{eqnarray}

where the gluon distribution in the rest frame of the
parton gas is defined as

\begin{equation}
  f(k^0;T)=\frac{\lambda_g}{e^{k^0/T}-1} \label{eq6}
\end{equation}
with $\lambda_g\leq 1$ specifying the deviation of the system from
chemical equilibrium. The relative velocity $v_{\rm rel}$ between
the $J/\psi$ and a gluon is
\begin{eqnarray}
  v_{\rm rel} &= &\frac{P_{J/\psi}\cdot k}{k^0M_T} \\
              &= &1-\frac{\vec{k}\cdot\vec{P}_T}{k^0M_T} \\
              &= &1-\frac{k P_T cos\theta }{k^0M_T} \\
              \label{eq7}
\end{eqnarray}

Changing the variable to the gluon momentum, $q=(q^0,\vec{q})$, in
the rest frame of the $J/\psi$, the integral in the numerator
of Eq.~(\ref{eq5}) can be rewritten as

\begin{equation}
  \int d^3q \frac{M_{J/\psi}}{M_T}\sigma(q^0) f(k^0;T), \label{eq8}
\end{equation}

\begin{equation} 
=\frac{M_{J/\psi}}{M_T} \int d^3q \sigma(q^0) \frac{\lambda}{e^{\frac{k^0}{T}}-1}, 
\label{eq11}
\end{equation}

where

\begin{equation}
  k^0=(q^0M_T+\vec{q}\cdot\vec{P}_T)/M_{J/\psi} \; . \label{eq9}
\end{equation}

putting the value of k$^{0}$ and solving

\begin{eqnarray} 
&= &\frac{M_{J/\psi}}{M_T} \int d^3q \sigma(q^0) \frac{\lambda_{g}} {  e^{ \frac{q^0m_{T}}{M_{J/\psi}T}} e^{ \frac{ \vec{q}\cdot \vec{p_{T} } }{ M_{J/\psi}T }  } -1}  \\
&= &\lambda_{g} \frac{M_{J/\psi}}{M_T} \int d^3q \sigma(q^0)   \sum_{n=1}^{\infty}  e^{ \frac{-n\,q^0m_{T}}{M_{J/\psi}T}} e^{  \frac{-n\,\vec{q}\cdot \vec{p_{T}}}{M_{J/\psi}T}}\\
&= &\lambda_{g} \frac{M_{J/\psi}}{M_T} \int\,q^2\,dq\,d(cos\theta)\,d\phi\,\sigma(q^0) \sum_{n=1}^{\infty}  e^{ \frac{-n\,q^0m_{T}}{M_{J/\psi}T}} e^{  \frac{-n\,\vec{q}\cdot \vec{p_{T}}}{M_{J/\psi}T}}\\
&= &\lambda_{g} \frac{M_{J/\psi}}{M_T}  \sum_{n=1}^{\infty} 2\pi  \int q^2 dq \, \sigma(q^0)\,e^{ \frac{-n\,q^0m_{T}}{M_{J/\psi}T}}  \int_{1}^{-1} e^{  \frac{-n\,q\,p_{T} cos\theta}{M_{J/\psi}T}} d(cos\theta)\\
&= &\lambda_{g} \frac{M_{J/\psi}}{M_T} \sum_{n=1}^{\infty} 2\pi  \int q^2 dq \, \sigma(q^0)\,e^{ \frac{-n\,q^0 m_{T}}{M_{J/\psi}T}} 
    [e^{-\frac{nq p_T}{M_{J/\psi}T}} - e^{\frac{nq P_T}{M_{J/\psi}T}}]\,\frac{M_{J/\psi}T}{nq p_T}\\
&= &\lambda_{g} \frac{M_{J/\psi}^2}{M_T} 2\pi \sum_{n=1}^{\infty} \frac{T}{n} \int_{\epsilon_0}^{\infty} q dq \, \sigma(q^0) \frac{1}{p_T} \,e^{ \frac{-n\,q^0 m_{T}}{M_{J/\psi}T}} 
    [e^{\frac{n q p_T}{M_{J/\psi}T}} - e^{- \frac{n q P_T}{M_{J/\psi}T}}]\\
\end{eqnarray}

$\sigma(q^0)$ can be written as

\begin{equation}
\sigma(q^{0}) = 4\pi\,(\frac{8}{3})^3\,\frac{1}{m_Q^{3/2}}\,\epsilon_0^3 \frac{ (q^0-\epsilon_0)^{3/2}}{(q^0)^5}
\end{equation}



One can carry out the integral in the denominator,
$\int d^3k f(k^0;T)=8\pi \zeta(3)\lambda_g T^3$, and the angular
part in the numerator, to get


\begin{equation}
  \langle v_{\rm rel}\sigma(k\cdot u)\rangle_k=
(\frac{8}{3})^3\frac{\pi}{\zeta(3)}\frac{M_{J/\psi}^2}{P_TM_T T^3}
(\frac{\epsilon_0}{m_Q})^{3/2}\sum_{n=1}^{\infty}T_n
\int_1^{\infty} dx\frac{(x-1)^{3/2}}{x^4}(e^{-a_n^-x}-e^{-a_n^+x})\; ,
\label{eq10}
\end{equation}


with $T_n=T/n$ and
\begin{equation}
  a_n^{\pm}=\frac{\epsilon_0}{T_n}\frac{M_T\pm P_T}{M_{J/\psi}}
  \; . \label{eq11}
\end{equation}


In order to understand the temperature and $P_T$ dependence of the
thermal gluon-$J/\psi$ dissociation cross section,
we first plot in Fig.~1 the cross section $\sigma(q^0)$ of
Eq.~(\ref{eq3}) as a function of the gluon energy in the $J/\psi$' rest frame.
It decreases strongly toward the threshold and is broadly
peaked around $q^0=10\epsilon_0/7=0.92$ GeV, with a maximum value
of about 3 mb. Low-momentum gluons have neither the resolution to
distinguish the heavy constituent quarks nor the energy to
excite them to the continuum. On the other end, high-momentum
gluons also have small cross section with a $J/\psi$ since they
cannot see the large size.


We can also express the cross section as a function of the
center-of-mass energy of gluons and the $J/\psi$,
$\sigma(q^0)=\sigma(s/2M_{J/\psi}-M_{J/\psi}/2)$, where
$s=(k+P_{J/\psi})^2$. One can thus translate the energy
dependence in Fig.~1 into temperature and $P_T$ dependences
after thermal average, since the thermally averaged $<s>$
is proportional to both $P_T$ and temperature $T$.
In Fig.~2 we plot the thermally averaged gluon-$J/\psi$ dissociation
cross section as a function of temperature for
different values of the $J/\psi$'s transverse momentum $P_T$.
We observe the same kind of peak structure, with a decreased
maximum value due to the thermal average. The position of the
peak also shifts to smaller values of $T$ when $P_T$ is
increased, corresponding to a fixed value of the averaged
center-of-mass energy $\langle s\rangle$. A similar behavior is
expected if one plots the thermal
cross section as a function of $P_T$ at different temperatures, as done
in Fig.~3. However, in this case, the peak simply disappears
at high enough temperatures, because the averaged $\langle s\rangle$
will be above the threshold value even for $P_T=0$.
These features will have considerable consequences for
the survival probability of a $J/\psi$ in an equilibrating
parton gas, especially the $P_T$ dependence.
We should also mention that the use of the Bose-Einstein
distribution function has an effect of about 20\% on the
thermal cross section, compared to that
obtained with a Boltzmann distribution \cite{ks95}.






%%%%%%%%%%%%%%%%%%%%%%%%%%%%%%%%%%%%%%%%%%%%%%%%%%%%%%%%%%%%%%%%%%%%%%%%%%%%%%%%%%%%
%%%%%%%%%%%%%%%%%%%%%%%%%%%%%%%%%%%%%%%%%%%%%%%%%%%%%%%%%%%%%%%%%%%%%%%%%%%%%%%%%%%%
\subsection{Formation Rate}
We can calculate Formation rate from Dissociation rate using detailed balance relation

\begin{equation}
 \sigma_{F} = \frac{48}{30}\,\sigma_{D}\frac{(s-M_{J/\psi})^{2}}{s(s-4m_{c}^{2})}
\end{equation}

or Formation rate can be written as 

\begin{equation}
\lambda_{F} = <\sigma_{F} \,\, v_{\rm relative}>
\end{equation}

v$_{\rm relative}$ is relative velocity between c $\bar{c}$ quark pair and can be defined as

\begin{equation}
v_{\rm relative} = {((p_{1}.p_{2})^{2} - m^{4} )^{1/2} \over E_{1}.E_{2}}
\end{equation}

\begin{equation}
=\frac{\sqrt{s(s-4m_{Q}^{2})}}{2E_1E_2} 
\end{equation}

%$m$ is reduced mass of system. 
%\begin{equation}
%m = { m_Q \,m_{\bar Q} \over m_Q + m_{\bar Q}} 
%={m_{Q}\over 2}
%\end{equation}
where 
\begin{eqnarray}
 s &= &(E_1+E_2)^{2} - (\vec{p_1}+\vec{p_2})^2 \\
   &= & m_1^{2} + m_2^{2} + 2 E_1E_2 - 2 |\vec{p_1}||\vec{p_2}|cos\theta \\
\end{eqnarray}


where $\vec{p_{1}}$ and $\vec{p_{2}}$ are three momentum of quarks. 

\begin{eqnarray}
\vec{p} &= &(p_x, p_y, p_z) \\
            &=&(p sin\theta cos \phi, p sin\theta sin \phi, p cos\theta)\\
\vec{p_1}.\vec{p_2} &= &|\vec{p_1}||\vec{p_2}|sin \theta_1 sin \theta_2 cos(\phi_1 - \phi_2) + cos \theta_1 cos\theta_2\\	
\end{eqnarray}


%We can write expression of v$_{relative}$ in
%\begin{equation}
%v_{\rm relative} = { [ p_1^{2} p_2^{2} ( sin\theta_1 sin\theta_2 cos (\phi_1 - \phi_2)  + cos\theta_1 cos \theta_2
% )^{2} - m^{4} ] ^{1/2} \over E_{1}E_{2}}
%\end{equation}

Total expression for formation rate can be written as

\begin{eqnarray}
\lambda_{F} &= &\frac{\int \sigma_{F}(s)\, v_{\rm relative}\,f_{c}(p_1)\, f_{\bar{c}} (p_2) \,d^{3}p_1 \,d^{3}p_2} {\int \,f_{c}(p_1)\,d^{3}p_1 \int f_{\bar{c}} (p_2)\,d^{3}p_2}\\
            &= &\frac{\int \sigma_{F}(s)\, v_{\rm relative}\,f_{c}(p_1)\, f_{\bar{c}} (p_2) 4\pi p_1^{2} dp_1 2\pi p_2^{2} dp_2 d(cos \theta)}
            {4\pi \int p_1^{2} dp_1 \,f_{c}(p_1) 4 \pi \int p_2^{2} dp_2 \, f_{\bar{c}} (p_2)}\\
\end{eqnarray}

here $\sigma_{F}$ is cross section of c$\bar{c}$ pair going to J/$\psi$ particle.
which is defined as
 \begin{equation}
\sigma_{F} ={ \theta(q^{2})\,\theta(4m_{D}^{2}-4m_{c}^{2}-q^{2})\,\sigma(J/\psi) \over N_{c\bar{c}} }
\end{equation}

here $\theta$(x) is Heaviside Step Function. We take $\sigma_{pp}$ = 7 fm$^{2}$ and
m$_D$ = 1.868 GeV.

and $f_{c}(p)$ and $f_{\bar{c}}(p)$ are parton distribution function of c and $\bar{c}$ respectively.

\begin{equation}
f_{c}(p)={g_c(=6)  \over 1+exp [ { (p^{2} + m^{2} )^{1/2}  \over T }] }
\end{equation}


and 
\begin{equation}
d^{3}p=p^{2} dp \, d(cos\theta)\,d\phi
=p^{2}dp\,sin\theta d\theta\,d\phi
\end{equation}


so we get a six dimensional integral which can be solved using Monte Carlo  methods. 




\section{Summary}
  The Table (I) summarizes all the above numbers. In conclusion at LHC all most all of the 
quarkonium prduced in the collisions will be suppressed. Since the number of initially 
produced charm pairs is very large it will give a large number of J/$\psi$ at hadronization.
The results from different models differ substantially. The 
transverse momentum distributions of these secondary  J/$\psi$ difffer from those 
initially produced. This aspect is being studied in detail. 


\noindent
\begin{thebibliography}{100}
\medskip
\bibitem{INTRO} I. Arsene {\it et al.} [BRAHMS Collaboration], Nucl. Phys. A {\bf 757}, 1 (2005); 
  B.B. Back {\it et al.} [PHOBOS Collaboration], Nucl. Phys. A {\bf 757} 28.(2005); 
  J. Adams {\it et al.} [STAR Collaboration], Nucl. Phys. A {\bf 757}, 10.(2005); 
  K. Adcox {\it et al.} [PHENIX Collaboration], Nucl. Phys. A {\bf 757} 184 (2005).
\bibitem{QGP_Tc} B. Muller, J. Schukraft and B. Wyslouch, Ann. Rev. Nucl. Part. Sci., arXiv:1202.3233 [hep-ex].
\bibitem{SATZ} T. Matsui and H. Satz, Phys. Lett. B{\bf 178}, 416 (1986).
\bibitem{ATLAS} G. Aad {\it et al.} [ATLAS Collaboration], Phys. Lett. B{\bf 697},294 (2011); arXiv:1012.5419.

\bibitem{JCMS} S. Chatrchyan {\it et al.} [CMS Collaboration]
 J. High Energy Phys. {\bf 1205}, 63 (2012);  arXiv: 1201.5069 [nucl-ex]..
\bibitem{CMSU2} CMS Collaboration, CERN-PH-EP-2012-228, arXiv:1208.2826.
\bibitem{UCMS} S. Chatrchyan {\it et al.} [CMS Collaboration] Phys. Rev. Lett. {\bf 107}, 052302 (2011).

\bibitem{YSuppAbdShuk} A. Abdulasalam, Prashant Shukla arXiv:1210.7584.

%Quarkonia
\bibitem{Vogt} R. Vogt, Phys. Rev. C{\bf 81}, 044903 (2010); arXiv:1003.3497.
\bibitem{Rapp1} X. Zhao and R. Rapp, Nucl. Phys. A{\bf 859}, 114 (2011); arXiv:1102.2194. 
\bibitem{Rapp2} X. Zhao and R. Rapp, Phys. Rev. C{\bf 82}, 064905 (2010); arXiv:1008.5328.

\bibitem{UPsi_Blaiz} J.P. Blaizot and J.Y. Ollitrault, Phys. Lett. B{\bf 199}, 499 (1987).
\bibitem{UPsi_Guni} J.F. Gunion and R. Vogt, Nucl. Phys. B{\bf 492}, 301 (1997).
\bibitem{CHU} M.C. Chu and T. Matsui, Phys. Rev. D{\bf 37}, 1851 (1988).


\bibitem{GAVIN} S. Gavin, P. L. McGaughey, P. V. Ruuskanen, and R. Vogt,
               Phys. Rev. C {\bf 54}, 2606 (1996).
               
\bibitem{LIN} Z. Lin, R. Vogt and X. N. Wang, Phys. Rev. C {\bf 57}, 899 (1998).
\bibitem{CTEQ6} J.~Pumplin, D.~R.~Stump, J.~Huston, H.~L.~Lai, P.~M.~Nadolsky 
and W.~K.~Tung,
  JHEP {\bf 0207}, 012 (2002)
  [arXiv:hep-ph/0201195];
  D.~Stump, J.~Huston, J.~Pumplin, W.~K.~Tung, H.~L.~Lai, S.~Kuhlmann 
  and J.~F.~Owens,
  JHEP {\bf 0310}, 046 (2003) 
  [arXiv:hep-ph/0303013].
\bibitem{EPS09} K. J. Eskola, H. Paukkunen and C. A. Salgado, JHEP
{\bf 0904}, 065 (2009) [arXiv:0902.4154 [hep-ph]].

\bibitem{CNV} M. Cacciari, P. Nason and R. Vogt, Phys. Rev. Lett.
{\bf 95}, 122001 (2005).

\bibitem{ContinuumVKShuk} V. Kumar, P. Shukla, R. Vogt arXiv:1205.3860.

\bibitem{MNR} M. L. Mangano, P. Nason, and G. Ridolfi, Nucl. Phys. B 
{\bf 373}, 295 (1992).


\bibitem{NVF} R. Nelson, R. Vogt and A. D. Frawley, submitted to Phys. Rev. C;
R. Vogt, R. E. Nelson and A. D. Frawley, arXiv:1207.6812 [hep-ph].
\bibitem{RVjoszo} R. Vogt, Eur. Phys. J.
Special Topics {\bf 155}, 213 (2008).
\bibitem{RVHP08} R. Vogt, Eur.\ Phys.\ J.\ C {\bf 61}, 793 (2009).


\bibitem{PbPbTotal} S. Chatrchyan {\it et. al.} (CMS Collaboration), 
Phys. Rev. C {\bf 84}, 024906 (2011).

\bibitem{MULT} K. Aamodt {\it et al.} [ALICE collaboration], Phys. Rev. Lett. {\bf 106}, 032301 (2011);
          arXiv:1012.1657 [nucl-ex].

\bibitem{CMSmult} S. Chatrchyan {\it et al.} [CMS Collaboration], J. High Energy Phys. {\bf 1108}, 141 (2011);
       arXiv:1107.4800.  

\bibitem{ks94}D.~Kharzeev and H.~Satz, Phys. Lett. {\bf B334}, 155(1994)

\bibitem{ks95}D.~Kharzeev and H.~Satz, CERN-TH/95-117, BI-TP 95/20,
         {\it Quark-Gluon Plasma II}, R. C. Hwa (Ed.) (World Scientific, Singapore)

\bibitem{KEXW}K.~J.~Eskola and X.-N.~Wang, Phys. Rev. D {\bf 49}, 1284(1994).


\bibitem{THEWS} R. L. Thews and J. Rafalski, Nuclear Physics A698, 575 (2002) [arXiv: hep-ph/0104025];
               R. L. Thews, arXiv: hep-ph/0206179.
\bibitem{MUNZI} P. Braun-Munzinger and J. Stachel, Physics Letters B490, 196 (2000).














%%%%%%%%%%%%%%%%%%%%%%%%%%%%%%%%%%%%%%%%%%%%%%
%\bibitem{NA50} M. C. Abreu {\it et al.} [NA50 collaboration], Nucl. Phys. A698, 127 (2002).

%\bibitem{yellow_report_qqb} CERN Yellow Report. “Hard probes in heavy ion 
 %collisions at the LHC: Heavy flavor physics”, CERN-2004-009-C; 
 %arXiv:hep-ph/0311048.

%\bibitem{DAIN} N. Carrer and A Dainese, arXiv:hep-ph/0311225.

%\bibitem{KHAR} D. Kharjeev, C. Lourenco, M. Nardi and H. Satz, Z. Phys. C 74, 307 (1997);
%[arXiv:hep-ph/9612217]. 


%\bibitem{BHAN} G. Bhanot and M.E. Peskin, Nucl. Phys B156, 391 (1979).

%\bibitem{MUNZI} P. Braun-Munzinger and J. Stachel, Physics Letters B490, 196 (2000).

%\bibitem{THEWS} R. L. Thews and J. Rafalski, Nuclear Physics A698, 575 (2002) [arXiv: hep-ph/0104025];
%               R. L. Thews, arXiv: hep-ph/0206179. 



%\bibitem{letessier} Rafelski, J., Letessier, J., and Tounsi, A., \emph{Heavy Ion Phys.}
%   181-192 (1996).

\end{thebibliography}

\end{document}
%
% ****** End of file snpcontrisamp.tex ******
