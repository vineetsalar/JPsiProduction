\documentclass[aps,prc,preprint,superscriptaddress,showpacs,showkeys]{revtex4-1}
\usepackage{graphicx}
\usepackage[usenames,dvipsnames,svgnames,table]{xcolor}

\begin{document}
{\Large \bf Commnets on ``Quarkonia suppression in PbPb collisions at $\sqrt s_{NN}$ =  2.76 TeV''}
%\author{\large Vineet Kumar}
%\author{\large Prashant Shukla}

%\maketitle
\bigskip
\bigskip
\bigskip
\bigskip

{\Large GENERAL PROBLEMS WITH MANUSCRIPT}

%\section{PROBLEMS WITH MANUSCRIPT}


\section{Figures}
In reviewing the figures of your paper, we note that the following
changes would be needed in order for your figures to conform to the
style of the Physical Review.  Please check all figures for the
following problems and make appropriate changes in the text of the
paper itself wherever needed for consistency.
\begin{itemize}
\item Figure(s) [2]
          Captions to color PostScript figures should begin with
          "(Color online)", unless they are to be published in
          color in print, in which case they should begin "(Color)".

\item Figure(s) [5-7]
          Figure sublabels should be printed on the figures.
          The preferred form is lowercase letters in parentheses:
          (a), (b), etc.

\item Figure(s) [5-7]
          Please increase the font size (axis labels, axis scale
          values, legend, etc.). Please ensure that all lettering is
          2 mm or larger (1.5 mm for superscripts and subscripts)
          after scaling to the final publication size. Note that
          the column width is 8.6 cm (twice that amount plus gutter
          for extra wide figures).
\end{itemize}

{\color{blue} All above suggestions are implemented.}

\section{REFERENCES}

Please remove the redundant arXiv or DOI references for published
papers. For your information, the editorial office checks
the references at several crucial steps during the editorial
process. A list of unnecessary arXiv references slows the process
down considerably--particularly so, if the list is long--and delays
the processing of your manuscript.

Please note that the copy editors will remove such redundant links
during production for those papers that have been accepted for
publication. However, any manual intervention carries the risk of
inadvertently introducing mistakes.

{\color{blue} Done}



\newpage


{\Large Report of the Referee -- CX10370/Kumar}
\bigskip
\bigskip
\bigskip
\bigskip

{\color{red} First of all, I would like to apologize to the authors for the delay
to produce this report.

This paper discusses the important topic of quarkonium (J/psi,
Upsilon, Upsilon') suppression in PbPb collisions at the LHC. The
suppression is calculated in a model which include various hot medium
effects (gluon dissociation, recombination, pion dissociation) in
addition to nuclear PDF corrections. The results are compared to ALICE
and CMS measurements.

I list below several comments and points which should be addressed
before I can recommend the paper to be published in Physical Review C.}

{\color{blue} We thank the referee for his suggestions. We have incorporated most 
of the suggestions. Some of them required considerable work to improve the manuscript. }


\section{Nuclear PDF (nPDF) effects}
%{\color{declared-color} some text}

\begin{enumerate}
\item { \color{red} The authors use only the central set of the EPS09 parametrization
instead of calculating the EPS09 uncertainty. Therefore it is
difficult for the reader to assess the uncertainty coming from nuclear
PDF corrections at the LHC. If it is numerically too difficult to
compute the EPS09, they could use for instance another nPDF set, e.g.
the recent DSSZ.}

{\color{blue} We include uncertanity due to all sets of EPS09 parametrization of nPDFs. This
uncertanity band is shown in figures with the central value of CNM effects.} 

\item { \color{red} There is little detail on how nPDF effects are computed. For
instance, it would be interesting to know what is the hard scale used
in the calculation. In particular, does it depend on pT? Also, it is
not clear to me how the heavy-quarkonium production cross section is
computed. I understand that heavy quark production is computed using
Refs. [26,27] but one needs a specific model in order to compute the
production of Q-Qbar bound states. Is it CEM?}

{\color{blue} 

%We take $\mu_F/m_T$ = $\mu_R/m_T$ = 1 as the central values for charm and bottom production. 
%Here $\mu_F$ is the factorization scale, $\mu_R$ is the renormalization scale, and
%$\sqrt{m_T = m^2 + p_T^2}$.
% We include uncertanity due to all sets of EPS09 parametrization of nPDFs [20].
%This uncertanity band is shown with the central value of CNM effects.
%   Heavy-quarkonium production cross section is computed using the Color Evaporation model [29]. 
%The Ref. [29] is newly added. 

% \begin{itemize}

%%\cite{Vogt:2010aa}
%\item [20]  R.~Vogt, ``Cold Nuclear Matter Effects on $J/\psi$ and $\Upsilon$ Production at the LHC,''
%  Phys.\ Rev.\ C {\bf 81}, 044903 (2010)
%  [arXiv:1003.3497 [hep-ph]]. 

%   %\cite{Nelson:2012bc}
% \item [29] R.~E.~Nelson, R.~Vogt and A.~D.~Frawley,
%   `Narrowing the uncertainty on the total charm cross section and its effect on the J/ψ cross section,''
%   Phys.\ Rev.\ C {\bf 87}, no. 1, 014908 (2013)
%   [arXiv:1210.4610 [hep-ph]].

%\end{itemize}

  We have recalculated all the cross sections presented in the table I using all new sets of PDFs 
  and parameters tuned to reproduce the measured data. We have taken help from Prof Ramona Vogt and hence 
  she has agreed to be author of the new version. The details are given in paper are presented as follows: 
  
  The heavy quark production cross section are calculated to NLO in pQCD  
  using the CT10 parton densities \cite{Lai:2010vv}. The mass and scale parameters used 
  for open and hidden heavy flavor production are obtained by fitting the energy dependence 
  of open heavy flavor production to the measured total cross sections~\cite{Nelson:2012bc,Nelson:Future}.
  Those obtained for open charm are $m_c = 1.27 \pm 0.09$~GeV,
  $\mu_F/m_{T\,c} = 2.10 ^{+2.55}_{-0.85}$, and $\mu_R/m_{T\, c} = 1.60 ^{+0.11}_{-0.12}$~\cite{Nelson:2012bc}. 
  The botttom quark mass and scale parameters are $m_b = 4.65 \pm 0.09$ GeV,
  $\mu_F/m_{T\, b} = 1.40^{+0.75}_{-0.47}$, and $\mu_R/m_{T\, b} = 1.10^{+0.26}_{-0.19}$~\cite{Nelson:Future}.
  The quarkonium production cross sections are calculated in the color evaporation model with
  normalizations determined from fitting the scale parameter to the shape of the energy-dependent
  cross sections~\cite{Nelson:2012bc,Nelson:Future}. The resulting uncertainty bands are smaller 
  than those obtained with the fiducial parameters used in Ref.~\cite{Cacciari:2005rk}.
  We note that the new results are within the uncertainties of those Ref.~\cite{Cacciari:2005rk}.  
  Indeed, the charm cross sections reported at the LHC agree
  better with the new values of the mass and scale than the central value of $m_c = 1.5$ GeV,
  $\mu_F/m_T = \mu_R/m_T = 1$


\bibitem{Lai:2010vv} 
  H.~L.~Lai, M.~Guzzi, J.~Huston, Z.~Li, P.~M.~Nadolsky, J.~Pumplin and C.-P.~Yuan,
  ``New parton distributions for collider physics,''
  Phys.\ Rev.\ D {\bf 82}, 074024 (2010).
  



\bibitem{Nelson:2012bc}
  R.~E.~Nelson, R.~Vogt and A.~D.~Frawley,
  `Narrowing the uncertainty on the total charm cross section and its effect on the J/$\psi$ cross section,''
  Phys.\ Rev.\ C {\bf 87}, no. 1, 014908 (2013).
  

  
\bibitem{Nelson:Future}
  R. Nelson, R. Vogt and A. D. Frawley, in preparation.
  
  
\bibitem{Cacciari:2005rk} 
  M.~Cacciari, P.~Nason and R.~Vogt,
  ``QCD predictions for charm and bottom production at RHIC,''
  Phys.\ Rev.\ Lett.\  {\bf 95}, 122001 (2005).
  %[hep-ph/0502203].



}

\item { \color{red} The nPDF effects seem to have a small but visible dependence on the
centrality of the collision, however the default EPS09 set has no
impact parameter (spatial) dependence. The authors should clarify the
origin of this Npart dependence. I am also quite surprised to see that
in the CMS kinematics, pT $\geq$ 6.5 GeV, the nPDF effects lead to an
enhancement (due to anti-shadowing), I was not expecting this because
of the rather small values of x$_{Bj}$ typically probed at the LHC.}

{
\color{blue} 
%We have to find some reasonable response for this comment. Probably we have to scale it
%with T$_{AA}$ instead of N$_{\rm Part}$

 The EPS09 set gives $p_T$ dependent shadowing factors and uncertainties. 
 To get $N_{\rm Part}$ dependence the $p_T$ integrated shadowing factor is linearly scaled by the 
variable $d$ (=$N_{\rm Coll}$ / ($N_{\rm Part}\times$0.5)) which mimics the density in the overlapping 
zone using minimum bias value and pp value which is 1.

%  $d \approx$ 6.4. 
 }


\item { \color{red} This is a detail but it is more appropriate to mention 'nuclear PDF
effects' instead of 'shadowing' which only applies to the small-x
depletion of the PDF in nuclei (one can see the effect of
anti-shadowing and the EMC effects in Fig. 5 left and Fig. 6 right).}
{\color{blue} We agree with it and in the figures we mention CNM Effects (Cold
Nuclear Matter Effects) and also change the title of section 2. }

\end{enumerate}


\section{Pion dissociation}

\begin{enumerate}

\item { \color{red} The authors use an arbitrary cross section of 1 mb independently of
the pion energy. They mention that this cross section is small,
however this is very similar in magnitude to the gluon-quarkonium
cross section, see Fig. 2. 
In order to have a consistent framework, the authors should rather use
the pQCD quarkonium-pion cross section, directly accessible from the
convolute the quarkonium-gluon cross section (Fig. 2) with the gluon
distribution in a pion, see Refs. [16,33,34] (see e.g. Ref. [34] in
which the quarkonium-pion cross section has a rapid variation with the
pion energy).}

{\color{blue} 
  Following referee's suggestion we have reworked this section dealing with hadronic comover.
  The pion-Quarkonia cross section is calculated by convoluting the gluon-quarkonia cross section $\sigma_D$
over the gluon distribution inside pion [38] as 

\begin{equation}
\sigma_{\pi Q} (p_{\pi}) = {p_+^2 \over 2(p_\pi^2 - m_\pi^2)} \int_0^1 \, dx \, G(x) \, \sigma_D(xp_+/\sqrt {2}) 
\end{equation}

where $p_+ = (p_\pi + \sqrt{p_\pi^2-m_\pi^2})/\sqrt{2}$. The gluon distribution $G(x)$ inside pion is 
given by GRV parameterization [42]. 


\begin{itemize}
% \item G. Bhanot and M. E. Peskin, “Short Distance Analysis for Heavy Quark Systems. 2. Applications,” Nucl. Phys. B 156, 391 (1979).
% \item  F. Karsch, M. T. Mehr and H. Satz, “Color Screening and Deconfinement for Bound States of Heavy Quarks,” Z. Phys. C 37, 617 (1988).
 \item [38] F. Arleo, P. B. Gossiaux, T. Gousset and J. Aichelin, Phys. Rev. D 65, 014005 (2002) [hep-ph/0102095].
 \item [42] M. Glueck, E. Reya, A. Vogt, Z. Phys. C{\bf 53} 651 (1992).
\end{itemize}

}

\item { \color{red} Apart from the energy dependence, the magnitude of the pion-J/$\psi$ and
pion-$\Upsilon$ cross section should not be the same, the latter being
smaller than the former by a factor 4-5.}

{\color{blue} 
%This will be automatically solved if we are able to use the p$_{T}$ dependent cross-section.
 This problem is taken care care when we use above procedure.
}

\item { \color{red} I fail to understand to understand why pion dissociation is much
less effective for $\Upsilon$ than for J/$\psi$ (compare Fig. 6 left and
Fig. 7 left), since the cross section is taken to be equal. Is it due
to the different p$_T$ spectrum?}

{ \color{blue} It was due to different $p_T$ spectra. But now the cross section for 
$\Upsilon$ is smaller as compared to J/$\psi$ due to tighter binding of $\Upsilon$. }

\item { \color{red} Also, what is assumed for the quarkonium(2S)-pion cross section? It
should be significantly larger than the quarkonium(1S)-pion cross
section. As a matter of fact, the comover effect seems more pronounced
on 2S states (compare Fig. 7 left and right) but nothing is said on
the comover effects on 2S in the model.}

{ \color{blue} The comover effect on $\Upsilon$(2S) is calculated as per above procedure now.


%It is mass effect. Mass of
%$\Upsilon$(2S) is significantly larger than $\Upsilon$(1S).
}

\end{enumerate}


\section{Hydrodynamical evolution}

\begin{enumerate}
\item { \color{red} The authors assume a transverse expansion of the medium, see Eq.
(8). In non-central collisions, however, the expansion should depend
on the azimuthal angle, leading to non-negligible elliptic flow. I
think the authors should comment on that.}

{ \color{blue} There will be azimuthal-anisotropy with respect to reaction plane. But this
effect will be averaged out over a large number of collisions.}

\end{enumerate}

\section{Comparison to data}

\begin{enumerate}

\item { \color{red} In order to have a meaningful comparison to data, I think the
authors should try to quantify the uncertainty of their calculation.
For instance, the most important effects in their model are gluon
dissociation and recombination, which crucially depends on the
quarkonium-gluon cross section. How would the predictions vary if they
change the magnitude of this cross section (which is poorly known) by,
say, a factor 1.5 or 2?}

{ \color{blue} 
%It can be implemented easily by hnad. We should check the effects on the final R$_{AA}$. We will check it by multipling by
%a factor of 1.5.
  The dominant sources of the uncretainties come from the gluon quarkonia cross section and initial temperature 
(discussed in next point). We vary the quarkonium-gluon cross section by a factor of 0.5 and 1.5 arround the
calculated value and obtain the variation in the final R$_{AA}$ calculations.   
}

\item { \color{red} Similarly, the magnitude on the medium energy density could be
varied in order to give a feeling to the reader on the corresponding
uncertainty on $R_{AA}$. Clearly the calculations are done for one set
of parameters and assumptions, it's difficult to know how the
predictions would change when those are varied.}


{ \color{blue} We calculate the initial temperature using 
measured charged particle density and assuming a value for $\tau_0$. 
We change it in reasonably large range from 0.1 to 0.6 fm/$c$ and the variation in  
$R_{AA}$ is calculated.
 Both of these uncertainties are added in quadrature to obtain the band. 

%nd see the effect. May be
%we can put a uncertanity band for this.
%Main perameters in our model are $\tau_0$, $T_C$ and some volume element 
%perameters. We can check the effects of these on the final results. Another thing may be to calculate the 
%rapidity dependent energy density by using rapidity dependent dN/d$\eta$. May be it does not make much sense because
%our calculations are in mid rapidity.

}

\item { \color{red} In this respect, many effects are included... but with uncertainty
which are not quantified. Since the effects of nPDF or comover affect
R$_{AA}$ by at most 20$\%$, I think it would make more sense not to include
them at all. In particular, the authors include effects with magnitude
10-20$\%$ while other cold nuclear matter effects such as energy loss in
nuclei prove much stronger (see e.g. 40$\%$ effects on J/psi in
arXiv:1407.5054) but are not included.}

{ \color{blue} 
  The uncertainites in the calculation are quantified. We still include CNM and comover effect.
  The prescriptions for change in the $R_{AA}$ of quarkonia due to initial energy loss in nuclei are 
still evolving we have not included this effect. It can be included in a future work. 
%Have to be seen in the reference. 
}


\item { \color{red} Also, the authors do not discuss the dissociation of
heavy-quarkonium due to the Debye screening of the heavy-quark
potential at finite temperature, which is historically the main effect
discussed in Ref. [1]. Do they consider it to be negligible? This
point would need to be discussed.}


{ \color{blue} The suppression of quarkonia can be understood in terms of 
gluon dissociation and/or color screening. 
  There is more colour screening at higher temperature. The gluon dissociation will also
be larger due to harder gluon momentum distribution.
  At higher temperature the gluon dissociation cross section can also increase due to colour 
screening and this effect must come in the uncertainty due to variation of the cross section.
}



\item { \color{red} Fig. 6 left and Fig. 7 left are a bit misleading because data points
corresponding to two different kinematical regions are included, but
only one calculation is provided and it is not possible to know which
region it corresponds to.}

{ \color{blue} 
  The calculations in this figure was for mid-rapidity. Now we have separated the forward 
rapidity and mid rapidity calculations in different figures.  
}

\item { \color{red} The authors present a model for $\Upsilon$(2S) and not for $\psi$(2S). 
I understand that for charmonia, and in particular 2S states, the pQCD
cross section may not be valid but still it would be very interesting
for the reader to know the model predictions. In particular, the CMS
experiment reported on a very interesting behavior of $\psi^{'}$/J/$\psi$ in
PbPb collisions, and it seems natural to wonder whether or not the
possibly large recombination in the $\psi^{'}$ channel could compare to
these data.}

{ \color{blue}   Since the $\psi$(2S) cross-section is not reliable so we have avoided to 
speculate on the charmonia ratio. }
\end{enumerate}


\section{Comparison to data}
\begin{enumerate}

\item { \color{red} Finally, many details or references are missing. In particular, why
R$_{0-5\%}$=0.92 R$_{Pb}$? What is the value of a$_m$? Which values are taken
for m$_c$ and m$_b$? It is also not clear to me why they need to compute
the entropy S(tau) in Eqs. (10,11), I probably missed the point.}

{\color{blue} The value of a$_m$ is 5 for QGP. %We use m$_c$ = 1.87 GeV and m$_b$ = 4.2 GeV 
%and section 2 is expanded to include these details. 
  R$_{0-5\%}$=0.96 R$_{\rm Pb}$ is obtained by  R$_{0-5\%}$ = R$_{\rm Pb}$ \, $\sqrt{N_{part,0-5}/N_{part,0}}$
where $N_{part,0}$ is number of participants in head on collision (b=0). The correct value is 0.96. 
 First we get the initial entropy for 0-5\% case from Eq. 11 and then use Eq. 10 to obtain entropy for
other centralities using the experimental measurement of ratio. 
}


\item { \color{red} On page 6, the reference to an experimental paper on dN/deta is
missing.}

{\color{blue} The Ref. 31 and 34 (new) are for ALICE. We have included the CMS reference [32] also.}


\item { \color{red} On Fig. 3, what is the rapidity of the quarkonium states? Is it
mid-rapidity? Maybe it would be more relevant to draw curves for a
given quarkonium energy E instead of binning in pT. It would also be
interesting to have the equivalent of Fig. 3 and 4 for 2S states.}

{\color{blue} 
Fig. 3 shows the calculations in mid rapidity. 
%This is also realted to the forward rapidity question. We can include the rapidity dependence by
%including y dependent gluon distribution and by putting p = p$_{\rm T}$ coshy in the formula of dissociation rate.
%see ref. arxiv:1005.1208 by authour U. Jameel and DKS.
%p$^{2}$ = p$_{\rm T}^{2}$ + m$_{\rm T}^{2} \cdot$sinhy
}

\end{enumerate}

{ \color{red} In conclusion, it would be valuable to address/answer these various
points in a revised version of the manuscript before I can recommend
its publication in Physical Review C.}

{ \color{blue} We have done considerable work to improve the manuscript so that almost all of the 
suggestions are incorporated. The revised manuscript is submitted.} 

\section{Additional improvements}

\begin{enumerate}



\item { \color{blue} We improved the calculations for the hevay flavour and quarkonia cross-sections. These calculations
  are improved by using CT10 PDFs also mass of charm and bottom quarks is tuned in such a way that it give
  better fitting to measured total charm production cross-section. Cross-sections and corresponding uncertainties 
  are shown in the table I.  }   

\item { \color{blue} Figures 3 (a) and 3(b) have same y axis ranges now similarly y axis ranges of Figures 4 (a) and 4(b) is 
  identical. }
  
%\item { \color{blue} We introduced a formula relating p$_{\pi}$ and center of mass energy $\sqrt{s}$ in section IV. } 
  
\item { \color{blue} We changes the order of figures. Now mid rapidity measurements of CMS and ALICE for J/$\psi$ nuclear modification
  factor is shown together in Figure 7(a) and 7(b). ALICE measurement in forward rapidity is shown separately in Figure 6.
  Similarly $\Upsilon$ measurements of CMS and ALICE are shown separately in Figure 8 and Figure 9 respectively.}        


\item { \color{blue} A final reading of the paper was done and at several places language, articles, punctuations
  have been improved. Wherever additional sentences are put they are highlighted in blue color.}


 
\end{enumerate}





\end{document}
