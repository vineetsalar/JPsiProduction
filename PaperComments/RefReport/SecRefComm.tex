\documentclass[aps,prc,preprint,superscriptaddress,showpacs,showkeys]{revtex4-1}
\usepackage{graphicx}
\usepackage[usenames,dvipsnames,svgnames,table]{xcolor}

\begin{document}
{\Large \bf Commnets on ``Quarkonia suppression in PbPb collisions at $\sqrt s_{NN}$ =  2.76 TeV''}
%\author{\large Vineet Kumar}
%\author{\large Prashant Shukla}

%\maketitle
\bigskip
\bigskip
\bigskip
\bigskip
{\color{red}
The authors have made a genuine effort to address most of the
(numerous) comments I made in my first report. In particular the
calculation of nPDF effects including theoretical uncertainties is now
properly done (within a given scheme) and many other clarifications
have been added. I deem the paper now suitable for publication in
Physical Review C.} \\

{ \color{black} Thanks }\\

{\color{red}
Let me add nevertheless a couple of additional comments, which the
authors could consider before publication} \\

\bigskip
\bigskip

{\color{red}
 I still think that there is some arbitrariness in adding this or
that effect in the 'cocktail' in order to compute J/$\psi$ and $\Upsilon$
suppression. For instance, the authors compute the dissociation by
pions which is an effect of the order of 5$\%$. At the same time, they
neglect other effects such as parton energy loss (in cold or hot
matter), which are likely to be significantly more important. Even
parton energy loss in cold nuclear matter, not included in the present
approach, is responsible for 40$\%$ for J/$\psi$ suppression in the most
central PbPb collisions (see 1407.5054). This would somehow change the
picture and spoil the agreement with the current measurements. This is
not discussed in the text}\\

{ \color{black}
  We discuss the effect of parton energy loss on quarkonia spectrum in the
result and discussion section of the manuscript. For reference we are quoting
the text below.\\ 
{ \color{blue}``The high $p_T$ suppression ($p_T > 10$  GeV/$c$) of J/$\psi$ 
measured by CMS is greater than that due to dissociation by gluons in the QGP. We note 
that at the highest $p_T$ values from CMS, $p_T \gg M_Q$, and energy loss might play a 
similar role for the $J/\psi$ at this $p_T$ as it does for open charm. So the 
large suppression observed in the high $p_T$ region may be due to energy loss inside the QGP.''}\\
Although the models describing effect of parton energy loss in the initial collisions 
on quarkonia are still evolving and it requires to be studied before we use them in our 
calculations. We plan to have such study in future calculations.     
}\\


{\color{red}
Related to the point above, I think it is better to rephrase the
legend of the figure. 'CNM effects' should be replaced by 'nPDF
effects' since other CNM effects are not taken into account. (Strictly
speaking mentioning 'nPDF effects' is even too general since this is
the effects of one specific nPDF set, EPS09, the most widely used in
the field.)
}\\


{ \color{black} 
We replace CNM effects to EPS09 NLO.
}\\


{\color{red}
The p$_T$ dependence of the J/$\psi$ R$_{AA}$ has been determined and compared
to data (Fig. 5). However, the equivalent suppression R$_{AA}(p_T)$ in the
Upsilon channel has not been computed, despite the fact that the Npart
dependence of Upsilon (1S and 2S) R$_{AA}$ is determined in Figs. 8 and 9.
I don't see any reason not to show this. In particular new CMS
preliminary data have been released a couple of weeks ago during the
Hard Probes conference, not to mention an earlier publication by CMS.
In the same spirit, I don't understand why the rapidity dependence of
J/psi and Upsilon suppression is not shown and compared to existing
data.
}\\


{ \color{black} 
  At the time of the submission of the manuscript CMS data for $\Upsilon$ , R$_{AA}(p_T)$
was having very large error bars. New data is recently shown in Hard Probe conference.
 We add one figure of Upsilon (1S) and (2S) pT dependent RAA. The 2S matches with the data 
and 1S has similar trend but remains above the data. It is because we do not include feeddown 
corrections important for 1S. We regret that we can not not put the data points on this figure. 
Because the data table is not public. As a co-author of CMS work we require to 
discuss it at CMS forum and that will delay this publication.
}\\




\begin{figure}
\includegraphics[width=0.49\textwidth]{Fig9a_CMS_Y1SRAAPt.pdf}
\includegraphics[width=0.49\textwidth]{Fig9b_CMS_Y2SRAAPt.pdf}
\caption{(Color online) Calculated nuclear modification factor ($R_{AA}$) as a function of $\Upsilon$ 
transverse momentum compared with CMS measurements.}
\label{fig:JPsiRaaVsPt}
\end{figure}


\end{document}






